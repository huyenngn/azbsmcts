\documentclass[12pt,oneside,openright]{article}
\usepackage{booktabs}
%Document Variables
\newcommand{\topic}
{An AlphaZero-inspired approach to imperfect information games}

\usepackage[utf8]{inputenc}
\usepackage[scaled]{helvet}
\renewcommand\familydefault{\sfdefault} 
\usepackage[T1]{fontenc}
\usepackage{fancyhdr,xcolor}
\usepackage{xcolor}
\usepackage{datetime2}
\usepackage[
    sorting=none,
]{biblatex}
\usepackage{float}
\usepackage{hyperref}
\usepackage{tabularx}
\hypersetup{
    colorlinks=false,
    linkcolor=black,
    filecolor=magenta,      
    urlcolor=cyan,
    pdftitle={Overleaf Example},
    pdfpagemode=FullScreen,
    }

\urlstyle{same}
\usepackage{graphicx}
\usepackage{geometry}
\geometry{
  a4paper,
  left=30mm,
  right=30mm,
  top=4.5cm,
  headheight=4cm,
  bottom=4.5cm,
  footskip=3cm
}
\usepackage{easyReview}

\renewcommand*{\bibfont}{\footnotesize}
\addbibresource{sample.bib}
\usepackage{xurl}
\newcommand{\changefont}{
    \fontsize{23}{26}\selectfont
}
\definecolor{boxcl}{HTML}{666666}
\definecolor{tubred}{HTML}{c50e1f}

\graphicspath{{assets/}}
\let\oldheadrule\headrule
\renewcommand{\headrule}{\color{tubred}\oldheadrule}
\renewcommand{\headrulewidth}{1.5pt}
\fancyfoot{}

\setlength{\parindent}{0pt}

\fancyhead[HL]{Bachelor's Thesis} 
\fancyhead[HR]{\includegraphics[width=0.15\textwidth]{assets/TUB.png}}
\fancyfoot[R]{\centering\thepage}
\pagestyle{fancy}
\begin{document}
\date{\today}
% \begin{titlepage}
\begin{center}
    \vspace*{1cm}
    \Huge
    \textbf{\topic}
    \LARGE
    %Thesis Subtitle

    \vspace{2cm}

    \textbf{Thi Nguyen Ngan Huyen}\\
    \vspace{0.5cm}
    \normalsize
    Matr Nr.: 400883\\
    E-mail: thinguyen@campus.tu-berlin.de
    \vspace{2cm}

    \large
    \begin{center}
        First Supervisor: Prof. Dr. Dr. h.c. Sahin Albayrak \\
        Second Supervisor: Dr.- Ing. Stefan Fricke
    \end{center}        \vspace{0.8cm}

    \Large
    Technische Universität Berlin\\
    Fakultät IV – Elektrotechnik und Informatik\\
    \vspace{1cm}
    \today
\end{center}
\newpage
% \end{titlepage}
\pagenumbering{arabic}

\section{Introduction}
5 - 8 pages

Purpose: Define the problem, why it matters, and what you actually do.

Contents:

Imperfect information in games (short, concrete)

Why Phantom Go is a good testbed

Why AlphaZero-style methods don’t directly apply

Your research question (one paragraph, no fluff)

Contributions (bullet list, explicit, modest)

Example contributions (realistic):

Implementation of BS-MCTS for Phantom Go

Design of an AlphaZero-inspired belief-based agent

Empirical comparison under controlled conditions

If you can’t state your contribution in 3 bullets, you don’t understand it yet.

\section{Background and Related Work}

10-15 pages

Purpose: Show you understand the landscape and position your work.

Split this cleanly:

2.1 Phantom Go and Imperfect Information

Rules of Phantom Go

Observation model (illegal moves, captures)

Why determinization is flawed (strategy fusion, non-locality)

2.2 Monte Carlo Tree Search Variants

Vanilla MCTS (brief)

IS-MCTS (why it fails)

BS-MCTS (high-level idea, defer algorithmic details)

2.3 AlphaZero

Core loop: self-play → MCTS → NN training

Why it assumes full observability

Known attempts at imperfect information (ReBeL, etc.)

Do not dump equations here. This section is conceptual.

\section{Problem Statement}

5 pages

Purpose: Make the problem mathematically and algorithmically precise.

This is where weak theses collapse. Yours must include:

Game definition (state, action, observation, belief)

What the agent actually observes

What “belief state” means in your implementation

What performance means (win rate, convergence, etc.)

If you skip this, your experiments will look arbitrary.

\section{Belief-State MCTS for Phantom Go}

12-15 pages

Purpose: This is your baseline and your anchor.

4.1 Belief-State Representation

How you represent sampled states

What is stored vs approximated

Memory and computation trade-offs

4.2 Algorithm Description

Walk through Algorithm 1 from the paper in your own words

Sampling

Selection (player vs opponent nodes)

Expansion, simulation, backpropagation

You must explain why opponent guessing/predicting exists, even if you simplify it.

4.3 Implementation Details

OpenSpiel integration

Key hyperparameters

Simplifications vs original paper (be honest)

This section alone can carry a bachelor’s thesis if done well.

\section{AlphaZero-Inspired Method for Phantom Go}

12-15 pages

Purpose: Show how AlphaZero breaks and how you fix it.

5.1 Why Vanilla AlphaZero Fails

Observation mismatch

Invalid state evaluation

Policy/value collapse

5.2 Belief-Based AlphaZero Architecture

Input representation (belief samples, aggregated features, etc.)

Modified MCTS (how it differs from Section 4)

Training targets (policy/value from belief-MCTS)

Be explicit: this is AlphaZero-inspired, not AlphaZero.

5.3 Training Procedure

Self-play setup

Network architecture (keep it simple)

Loss functions

Compute constraints

If you try to be “clever” here, you’ll fail. Simplicity wins.

\section{Experimental Evaluation}

10-12 pages

Purpose: This is where your thesis earns its grade.

6.1 Experimental Setup

Hardware

Board size

Time limits

Number of games

6.2 Baselines

BS-MCTS (from Section 4)

Possibly IS-MCTS (optional)

6.3 Results

Win rate comparisons

Learning curves

Stability vs compute

Sensitivity to hyperparameters

No storytelling. Tables, plots, facts.

\section{Discussion}

5–7 pages

Purpose: Show you can think critically.

Why AlphaZero-style methods help (or don’t)

Where belief modeling breaks

What limits performance (compute, belief noise, training instability)

Comparison to expectations from literature

This is where you admit weaknesses intelligently.

\section{Conclusion and Future Work}

3–5 pages

Purpose: Close the loop.

Answer the research question directly

Summarize what worked and what didn’t

Concrete future work (not vague “more experiments”)

\section{References}
\printbibliography[heading=none]

\end{document}
